\documentclass{article}
\usepackage[utf8]{inputenc}
\usepackage[spanish]{babel}
\usepackage{listings}
\usepackage{graphicx}
\graphicspath{ {images/} }
\usepackage{cite}

\begin{document}

\begin{titlepage}
    \begin{center}
        \vspace*{1cm}
            
        \Huge
        \textbf{Parcial 1 - Calistenia.}
            
        \vspace{0.5cm}
        \LARGE
            
        \vspace{1.5cm}
            
        \textbf{Luis Miguel Gil Rodriguez.}
            
        \vfill
            
        \vspace{0.8cm}
            
        \Large
        Despartamento de Ingeniería Electrónica y Telecomunicaciones\\
        Universidad de Antioquia\\
        Medellín\\
        Marzo de 2021
            
    \end{center}
\end{titlepage}

\tableofcontents
\newpage
\section{Sección introductoria}\label{intro}
Esta es la primera sección, podemos agregar algunos elementos adicionales y todo será escrito correctamente. Más aún, si una palabra es demasiado larga y tiene que ser truncada, babel tratará de truncarla correctamente dependiendo del idioma.

\section{Intrucciones a seguir.} \label{contenido}
Esta sección es para agregar toda la información correspondiente con código, citas, etc.

\section{Conclusiones.} \label{imagenes}
Es probable que muchas personas no logren ver la importancia de este experimento, pero ese no mi caso; hablando desde el ámbito de la programación, esta sencilla prueba es una ayuda para comprender que tan fundamental es la buena codificación de un programa, siendo lo menos ambiguo posible, sin caer en redundancias o peor aun en contradicciones. Todo esto será de útil ayuda para lograr un bien nivel de lógica de programación, entender y codificar de una manera más eficiente.
\\
\\
Ahora bien, fuera del ámbito de la programación, este experimento también cobra importancia, no nos alejemos tanto de nuestra situación y pensemos ¿qué sería de un profesor que no se supiera como explicar bien los temas?, es muy probable que la gran mayoría de sus estudiantes terminen cancelando el curso. Ahora, pensemos a futuro, ¿qué sería de un jefe que no sepa cómo dar instrucciones precisas y concisas?, es muy probable que todo el proyecto que en el que se encuentran trabajando salga mal o bien tarde más tiempo de lo esperado. Por lo que una correcta comunicación e impartición de instrucciones es fundamental para poder llevar a cabo un proyecto en grupo.
\end{document}
